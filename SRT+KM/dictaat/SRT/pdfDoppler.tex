\chapter{Het Doppler-Effect}

%%%%%%%%%%%%%%%%%
%\section{Het klassieke Doppler-effect}
%% Nodig: Volledige inleiding Doppler-effect!!

\section{Relativistische afleiding}
Het Doppler-effect kent iedereen uit het dagelijks leven. Wanneer een
politie-auto met sirene aan langs ons rijdt, dan klinkt de toon van de
sirene hoger als de auto naar ons toe rijdt en lager wanneer de auto
zich van ons af beweegt.  Het Doppler-effect is een elementair
verschijnsel dat optreedt bij alle trillingen die zich voortplanten.

\subsection{De golfvergelijking}
Een willekeurige golf die zich voortplant in de 1-dimensionale ruimte
wordt beschreven door de algemene formule
%
\begin{equation} 
\varphi(x,t) = A \sin(kx - \omega t) + B \cos(kx -\omega t) .
\end{equation}
%
Dit geldt voor alle soorten golven: water-, geluids- of lichtgolven,
zolang ze opgebouwd zijn uit \'e\'en enkele frequentie.

De variabelen $k$ is het zogenaamde golfgetal, $\omega$ is de
cirkelfrequentie. De constanten $A$ en $B$ bepalen de amplitude en
beginconditie van de golf. Als we simpelweg aannemen dat op $t=0$ en
$x = 0$ geldt $\varphi(0,0) = 0$, dan is $B = 0$. We hebben dan
%
\begin{equation} \label{golfVerg}
\varphi(x,t) = A \sin(kx -\omega t).
\end{equation}

Om deze formule verder te bekijken nemen we eerst een vast tijdstip $t
= T$. De maxima van $\varphi$ worden gegeven door
%
\begin{equation} \label{maxima}
kx - \omega T = (2n + \frac{1}{2}) \pi, \qquad n = 0,1,2,\ldots
\end{equation}
De afstand tussen twee op elkaar volgende maxima is per definitie de
golflengte en geven we aan met $\lambda$. Vergelijk in formule
(\ref{maxima}) de gevallen $n = 0$ en $n = 1$ en we zien dat de
golflengte $\lambda = 2\pi / |k|$.

Als we vervolgens formule (\ref{golfVerg}) voor een vaste $x$ bekijken
zien we dat de tijd tussen twee op elkaar volgende maxima gelijk is
aan $2\pi / \omega$. Dit betekent dat de frequentie $\nu$ gelijk is
aan het omgekeerde, $\nu = \omega / 2\pi$. Combinatie van golflengte
en frequentie levert ons nu op dat de snelheid $v_g$ waarmee een
maximum beweegt, de voortplantingssnelheid, gelijk is aan $ \omega /
k$. Voor de absolute waarde van $v_g$ geldt dan $|v_g| = \nu \lambda$.

Een 1-dimensionale golf wordt dus in het algemeen bepaald door de
constanten $A$ en $B$, en door drie parameters $\lambda$, $\nu$ en
$v_g$, waarvan er twee onafhankelijk zijn:
%
\begin{equation}
\lambda = 2\pi / |k|, \qquad \nu = \omega / (2\pi), \qquad v_g = \omega / k = \pm \nu \lambda.
\end{equation}

\subsection{Lorentztransformaties}

We veronderstellen nu dat de waarde die de amplitude $\varphi$ op een
zeker moment op een bepaalde plaats aanneemt onafhankelijk is van het
inertiaalsysteem waarin we ons bevinden.\footnote{Dit is niet altijd correct: electromagnetische golven
transformeren niet triviaal onder een Lorentztransformatie. Voor nu
kunnen we dit echter verwaarlozen.}  We kunnen dan met behulp van een
Lorentztransformatie overgaan van het gegeven systeem $S$, met
co\"ordinaten $x$ en $t$, naar een nieuw systeem $S'$, met $x'$ en
$t'$. In de nieuwe co\"ordinaten wordt de golf beschreven door een
nieuwe golffunctie
%
\begin{equation} \label{phi'=phi}
\varphi'(x', t') = \varphi(x, t).
\end{equation}
%
Met de formule (\ref{v:lorentz12}) voor de inverse Lorentztransformatie schrijven we 
%
\begin{equation} \label{phinaarphi'}
kx - \omega t = k [\gamma (x' + \beta c t')] - \omega [\gamma (t' + \frac{\beta}{c} x')].
\end{equation}
%
In systeem $S'$ heeft de golfbeweging nog steeds dezelfde vorm, maar nu  met golfgetal $k'$ en cirkelfrequentie $\omega'$. Daarmee wordt de uitdrukking voor $\varphi'$ gelijk aan 
%
\begin{equation}
\varphi'(x',t') = A \sin(k'x' - \omega't')
\end{equation}
%
Combineren we dit nu met formule (\ref{phi'=phi}) en (\ref{phinaarphi'}) dan zien we dat 
%
\begin{equation}
k' = \gamma (k - \omega \frac{\beta}{c}) , \qquad \omega' = \gamma (\omega - \beta c k).
\end{equation}
%

We veronderstellen nu dat we te maken hebben met een electromagnetische golf, dus een lichtgolf of een radiogolf, die zich in het vacu\"um voortplant. We nemen $k > 0$. Uit $|\beta| < c$ volgt $k' > 0$. Voor electromagnetische golven geldt $\omega = c|k|$, dus hier $\omega = ck$. Voor $k'$ hebben we dan 
%
\begin{equation}
k' = \gamma (k - \omega \frac{\beta}{c}) = \gamma k (1 - \beta) = k \sqrt{\frac{1 - \beta}{1 + \beta}}
\end{equation}
%
en voor $\omega'$
%
\begin{equation}
\omega' = \gamma (\omega - \beta c k) = \gamma \omega (1 - \beta) = \omega \sqrt{\frac{1 - \beta}{1 + \beta}} .
\end{equation}
%
Voor de gewone frequentie $\nu$ vinden we op deze wijze 
%
\begin{equation}
\nu' = \frac{\omega'}{2\pi} = \frac{\omega}{2\pi}\sqrt{\frac{1 - \beta}{1 + \beta}} = \nu \sqrt{\frac{1 - \beta}{1 + \beta}}.
\end{equation}
%

Voor $\beta > 0$ ziet een waarnemer in $S'$ het systeem $S$ van zich af bewegen. Voor $\beta < 0$ ziet hij $S$ juist naar zich toe bewegen. Volgens de relativiteitstheorie wordt dus licht met frequentie $\nu$, dat wordt uitgezonden door een lichtbron die zich van ons af beweegt, door ons waargenomen als licht met de verlaagde frequentie
%
\begin{equation} \label{nuDoppEff}
\nu' = \nu  \sqrt{\frac{1 - \beta}{1 + \beta}}.
\end{equation}
%
Beweegt de lichtbron naar ons toe, dan zien we de verhoogde frequentie 
%
\begin{equation} \label{e:doppler}
\nu' = \nu  \sqrt{\frac{1 + |\beta|}{1 - |\beta|}}.
\end{equation}
%

\subsection{De niet-relativistische limiet}

In de alledaagse, niet-relativistische fysica treedt het
Doppler-effect ook regelmatig op. De verandering in waargenomen
frequentie is dan nog steeds te berekenen met behulp van formule
(\ref{nuDoppEff}), maar we kunnen ook kijken wat het resultaat is als
we de Galileitransformaties gebruiken in plaats van de Lorentztransformaties.

We gaan opnieuw uit van een enkele golf gegeven door de formule
(\ref{golfVerg}). Bekijken we deze vanuit de twee inertiaalsystemen
$S$ en $S'$ dan wordt het verband tussen deze systemen nu gegeven door
de Galileitransformatie (\ref{v:galilei1b}). Met behulp van de
inverse van deze transformatie vinden we in plaats van formule
(\ref{phinaarphi'})
%
\begin{equation}
kx - \omega t = k(x' + vt') - \omega t' = kx' - (\omega - kv)t' .
\end{equation}
%
Het golfgetal $k$ verandert niet en we vinden voor de nieuwe cirkelfrequentie 
%
\begin{equation}
\omega' = \omega - kv .
\end{equation}
%
Voor een electromagnetische golf met $k > 0$ wordt dit
%
\begin{equation}
\omega' = \omega (1 - \beta) .
\end{equation}
%
Het niet-relativistische analogon van formule (\ref{nuDoppEff}) is dus 
%
\begin{equation} \label{nuDoppEffClass}
\nu' = \nu (1 - \beta) .
\end{equation}
%

We kunnen nu de relativistische en niet-relativistische formules voor het Doppler-effect vergelijken. Als $v$ klein is ten opzichte van $c$ kunnen we de relativistische formule (\ref{nuDoppEff}) ontwikkelen naar machten van $\beta$.
\footnote{ Dit is een zogenaamde Taylor-expansie: iedere functie $f(x)$
kan rond het punt $x = a$ worden geschreven als
\[
f(x) = f(a) + f'(a) (x - a) + \frac{f''(a)}{2!} (x - a)^2 + \frac{f'''(a)}{3!} (x - a)^3 + \ldots 
\]
Voor een wortelfunctie met $0 \ll \alpha \ll 1$ geldt er dan
\[
\sqrt{1 - \alpha} = (1 - \alpha)^{1/2} = 1 + \frac{1}{2} \alpha - \frac{1}{4} \alpha^2 + \ldots...
\]
}

We krijgen dan 
%
\begin{equation} \label{OntwDopp}
\nu' = \nu \sqrt{\frac{1 - \beta}{1 + \beta}} = \nu \sqrt{1 - \frac{2 \beta}{1 + \beta}} = \nu( 1 - \beta + \frac{1}{2} \beta^2 + \ldots) .
\end{equation}
%
Bij normale snelheden met $\beta \ll 1$ kunnen we de laatste term in
deze vergelijking verwaarlozen en we zien dat we precies de
niet-relativistische limiet bereiken van formule
(\ref{nuDoppEffClass}): het Doppler-effect voor licht bij normale
snelheden is zeer klein, dit in tegenstelling tot wat we bij
geluidsgolven waarnemen.

\section{Nogmaals het Doppler-effect}
We hadden gezien dat wanneer een geluidsbron zich naar ons toe beweegt, de toonhoogte zal toenemen
in vergelijking met de situatie waarin de bron in rust is.
We zullen nu nogmaals het relativistische Doppler-effect laten zien.
We hebben gezien dat voor licht per definitie de relatie $E = pc$ moet
gelden: deze relatie geldt immers als $v = c$.
De Lorentztransformatie voor een lichtstraal die zich in de $x$-richting
beweegt
ziet er dan als volgt uit (ga na):
\begin{displaymath}
E' = \gamma (1 - \beta) E
\end{displaymath}
Dit kunnen we ook schrijven als:
\begin{displaymath}
E' = \sqrt{\frac{1 - \beta}{1 + \beta}}E
\end{displaymath}
D.w.z. in co\"{o}rdinatenstelsel $S'$ dat met 
snelheid $\beta c$ in de richting van de lichtstraal
meebeweegt is de energie van de lichtstraal kleiner (het licht is verschoven
naar het rood) met een factor
$\sqrt{\frac{1 - \beta}{1 + \beta}}$,
Omgekeerd, bewegen we met snelheid $\beta c$ naar een lichtbron toe, dan
neemt de energie van het
licht toe met een factor $\sqrt{\frac{1 + \beta}{1 - \beta}}$
(het licht is verschoven naar het violet).
Met behulp van de relatie $E = h \nu$ (de energie van een lichtquantum, een
foton, van licht met frequentie $\nu$; $h$ is een natuurconstante, de
constante van Planck) volgt:
\begin{displaymath}
\nu' =  \sqrt{\frac{1 - \beta}{1 + \beta}}\nu
\end{displaymath}
precies zoals we ook eerder zagen in vergelijking~\ref{e:doppler}.

\section{Roodverschuiving en het uitdijende heelal}
Het licht uitgezonden door de vele duizenden melkwegstelsels in het heelal
blijkt verschoven te zijn naar lagere golflengtes (`roodverschuiving') t.o.v.
wat verwacht zou worden op basis van standaard spectra.
M.a.w. deze melkwegstelsels verwijderen zich van onze eigen Melkweg, met een
snelheid die te bepalen is m.b.v. bovenstaande formule voor relativistische
Dopplerverschuiving.
Het heelal dijt uit!
Deze observatie was een belangrijke bron van inspiratie voor de `Big Bang'
theorie van het ontstaan van het heelal.

