\subsection{Real-World Relativity: The GPS Navigation System} 
{\it Richard W. Pogge, 2004}
 
People often ask me "What good is Relativity?" It is a commonplace to think of Relativity as an abstract and highly arcane mathematical theory that has no consequences for everyday life. This is in fact far from the truth. 

Consider for a moment that when you are riding in a commercial airliner, the pilot and crew are navigating to your destination with the aid of the Global Positioning System (GPS). Further, many luxury cars now come with built-in navigation systems that include GPS receivers with digital maps, and you can purchase hand-held GPS navigation units that will give you your position on the Earth (latitude, longitude, and altitude) to an accuracy of 5 to 10 meters that weigh only a few ounces and cost around $100. 

GPS was developed by the United States Department of Defense to provide a satellite-based navigation system for the U.S. military. It was later put under joint DoD and Department of Transportation control to provide for both military and civilian navigation uses. 

The current GPS configuration consists of a network of 24 satellites in high orbits around the Earth. Each satellite in the GPS constellation orbits at an altitude of about 20,000 km from the ground, and has an orbital speed of about 14,000 km/hour (the orbital period is roughly 12 hours - contrary to popular belief, GPS satellites are not in geosynchronous or geostationary orbits). The satellite orbits are distributed so that at least 4 satellites are always visible from any point on the Earth at any given instant (with up to 12 visible at one time). Each satellite carries with it an atomic clock that "ticks" with an accuracy of 1 nanosecond (1 billionth of a second). A GPS receiver in an airplane determines its current position and heading by comparing the time signals it receives from a number of the GPS satellites (usually 6 to 12) and triangulating on the known positions of each satellite. The precision is phenomenal: even a simple hand-held GPS receiver can determine your absolute position on the surface of the Earth to within 5 to 10 meters in only a few seconds (with differential techiques that compare two nearby receivers, precisions of order centimeters or millimeters in relative position are often obtained in under an hour or so). A GPS receiver in a car can give accurate readings of position, speed, and heading in real-time! 

To achieve this level of precision, the clock ticks from the GPS satellites must be known to an accuracy of 20-30 nanoseconds. However, because the satellites are constantly moving relative to observers on the Earth, effects predicted by the Special and General theories of Relativity must be taken into account to achieve the desired 20-30 nanosecond accuracy. 

Because an observer on the ground sees the satellites in motion relative to them, Special Relativity predicts that we should see their clocks ticking more slowly. Special Relativity predicts that the on-board atomic clocks on the satellites should fall behind clocks on the ground by about 7 microseconds per day because of the slower ticking rate due to the time dilation effect of their relative motion. 

Further, the satellites are in high orbits, where the curvature of spacetime due to the Earth's mass is less than it is at the Earth's surface. A prediction of General Relativity is that clocks closer to a massive object will seem to tick more slowly than those located further away. As such, when viewed from the surface of the Earth, the clocks on the satellites appear to be ticking faster than identical clocks on the ground. A calculation using General Relativity predicts that the clocks in each GPS satellite should get ahead of ground-based clocks by 45 microseconds per day. 

The combination of these two relativitic effects means that the clocks on-board each satellite should tick faster than identical clocks on the ground by about 38 microseconds per day (45-7=38)! This sounds small, but the high-precision required of the GPS system requires nanosecond accuracy, and 38 microseconds is 38,000 nanoseconds. If these effects were not properly taken into account, a navigational fix based on the GPS constellation would be false after only 2 minutes, and errors in global positions would continue to accumulate at a rate of about 10 kilometers each day! The whole system would be utterly worthless for navigation in a very short time. This kind of accumulated error is akin to measuring my location while standing on my front porch in Columbus, Ohio one day, and then making the same measurement a week later and having my GPS receiver tell me that my porch and I are currently about 5000 meters in the air somewhere over Detroit. 

The engineers who designed the GPS system included these relativistic effects when they designed and deployed the system. For example, to counteract the General Relativistic effect once on orbit, they slowed down the ticking frequency of the atomic clocks before they were launched so that once they were in their proper orbit stations their clocks would appear to tick at the correct rate as compared to the reference atomic clocks at the GPS ground stations. Further, each GPS receiver has built into it a microcomputer that (among other things) performs the necessary relativistic calculations when determining the user's location. 

Relativity is not just some abstract mathematical theory: understanding it is absolutely essential for our global navigation system to work properly! 

