\chapter{Waarom relativiteit?} 
\vspace{-1cm}\begin{flushright}
{\it `Everything should be made as simple as possible,\\ 
but not simpler'} \\ A. Einstein
\end{flushright}
%\pagenumbering{arabic} [3]
Het jaar 2005 werd uitgeroepen tot het ``World Year of Physics''. Het was
precies 100 jaar geleden dat Einstein een aantal artikelen
publiceerde die ingrijpende vernieuwingen in de natuurkunde
teweegbrachten. Zo is hij met zijn publicatie over het
deeltjeskarakter van licht de grondlegger van de kwantummechanica. Een
ander grensverleggend artikel uit dat jaar is {\it Zur Elektrodynamik
bewegter K\"{o}rper - Over de elektrodynamica van bewegende lichamen}
en bevat de Speciale Relativiteitstheorie, en vormt het onderwerp van
dit college. Een derde artikel uit 1905 beschrijft de Brownse beweging
waarmee het bestaan van atomen feitelijk werd aangetoond. In Duitsland
werd het jaar 2005 daarom ook simpelweg het ``Einstein Jahr 2005''
genoemd.

Rond 1900 bestond er een fysisch probleem waar natuurkundigen als
 Lorentz en Poincar\'e zich het hoofd over braken. Er was met
 nauwkeurige metingen aangetoond dat de voortplantingssnelheid van het
 licht onafhankelijk is van de bewegingssnelheid van de waarnemer, en dit
 is niet in overeenstemming met de Klassieke Mechanica zoals opgesteld
 door Galilei en Newton.  Een volledige oplossing van dit probleem
 werd gegeven door Einstein's Special Relativiteitstheorie, waarmee
 hij fundamenteel nieuwe opvattingen gaf aan de begrippen ruimte en
 tijd, en het verband tussen die twee. Een aantal jaren later, in 1912,
 formuleerde hij de Algemene Relativiteitstheorie, waarin hij deze
 idee\"en verder uitbreidde en daarmee een nieuwe beschrijving van de
 zwaartekracht ontwikkelde.

De toepassingen van de Speciale Relativiteitstheorie in het dagelijks
leven zijn beperkt, omdat de gevolgen pas goed zichtbaar worden bij
zeer hoge snelheden, in de buurt van de lichtsnelheid, of bij zeer
nauwkeurige tijdsmetingen. De meeste mechanische verschijnselen om ons
heen worden gewoon prima beschreven door de Klassiek Mechanica, zoals
opgesteld door Newton. We zouden op deze alledaagse verschijnselen
ook de relativiteitstheorie los kunnen laten (want deze geldt voor
{\it alle} verschijnselen voor zover we weten), maar dit levert geen
voordeel op; het zou de beschrijving alleen nodeloos ingewikkeld
maken. Pas bij verschijnselen die bijvoorbeeld optreden in een
deeltjesversneller, waarbij zeer hoge snelheden bereikt worden, of bij de beschrijving van het
GPS navigatiesysteem waarbij heel nauwkeurige tijdmeting cruciaal is,
zou de Newtonse mechanica verkeerde resultaten opleveren. De `nieuwe'
relativiteitstheorie heeft dus een groter toepassingsgebied, maar de
oude theorie behoudt haar waarde voor een kleiner gebied van
verschijnselen waar ze goed gebruikt kan worden als een zeer goede
benadering van de algemenere theorie.

Behalve de fysisch waarneembare gevolgen heeft de relativiteitstheorie
vanaf het begin grote aantrekkingskracht uitgeoefend door de fraaie en
verrassende algemene inzichten. De Algemene Relativiteitstheorie
vereist wel de nodige wiskundige kennis, al zijn er aspecten die wel
eenvoudig te begrijpen zijn. Voor een goed begrip van de Speciale
Relativiteitstheorie is geen ingewikkelde wiskunde nodig. Door middel
van een aantal `gedachtenexperimenten', die veelal van verbluffende
eenvoud zijn, komen we tot de kern van deze theorie. Hiermee wordt een
compleet nieuw inzicht over met name de begrippen ruimte en tijd
verkregen.

Zich baserend op twee postulaten vond Einstein het
 correcte kader voor de hele mechanica, de elektrodynamica en, zo is
 onze vaste overtuiging, voor welke natuurkundige theorie dan ook, de
 meest moderne theorie\"{e}n inbegrepen. Dit kader, de Speciale
 Relativiteitstheorie, is het onderwerp van dit college.

\section{Tijd, ruimte en beweging} 
\begin{flushright}
{\it `I do not define time, space, place and motion, since they are \\ well known to all'} \\ Newton, Principia (1686)
\end{flushright}
We kennen allemaal het begrip `beweging', maar wat zou het antwoord
zijn als gevraagd wordt dit begrip te {\it defini\"eren}? Waarschijnlijk
zult u een formulering geven waarin staat ``verandering van positie in
de tijd'', of iets dergelijks. Want het begrip beweging is
onlosmakelijk verbonden met de concepten `ruimte' en `tijd'. We zeggen
dat een voorwerp beweegt als het verschillende posities inneemt op
verschillende tijdstippen.

De begrippen ruimte en tijd in de Klassieke Mechanica komen goed
overeen met de alledaagse intu\"itie. Hieronder worden, in het licht
van de Klassieke Mechanica, deze begrippen nader beschreven. Let op!
We zullen deze begrippen in dit college grondig herzien - deze
beschrijving moet met kritische blik worden gelezen:
\begin{verse}
{\it Ruimte} in de ogen van Newton is absoluut, in de zin dat het
permanent bestaat en onafhankelijk is van materie die erin
beweegt. Newton zegt hierover in de Principia: ``Absolute space, in
its own nature, without relation to anything external, remains always
similar and immovable''.
\end{verse}
Ruimte is dus een soort drie-dimensionale matrix waarin men naar
believen voorwerpen kan plaatsen die kunnen bewegen, en er is geen
interactie tussen de voorwerpen en de ruimte. Elk voorwerp in het
universum bestaat op een bepaalde plaats in de ruimte en tijd. Een
voorwerp in beweging ondergaat een continue verandering van plaats in
de tijd. En hoewel niet praktisch, is het mogelijk alle posities in
kaart te brengen met behulp van een groot netwerk van meetstokken,
uitgelegd in de drie-dimensionale, kubische ruimte. Met andere woorden:
de ruimte is er, en we kunnen in principe elk punt markeren. Alle
natuurkundige experimenten stemmen overeen met de theorema's van
Euclides' geometrie, en ruimte is daarmee Euclidisch.
\begin{verse}
{\it Tijd}, in Newtons beleving, is ook absoluut en tikt met ijzeren
regelmaat. Weer in de Principia, stelt Newton: ``Absolute, true, and
mathematical time, of itself, and from its own nature, flows equably
without relation to anything external, and by another another name is
called duration''. 
\end{verse}
De taal is mooi, maar de verklaring van Newton is niet erg
informatief. Zoals in de quote boven deze paragraaf te lezen is,
probeerde Newton ook niet echt ruimte en tijd te defini\"eren.  Tijd kan
niet versneld of vertraagd worden, en de tijd tikt uniform in het
hele universum.  We kunnen ons het `nu' voorstellen zoals het
gelijktijdig plaatsvindt op elke planeet en ster in het heelal. Een uur
tijdsverschil is identiek voor elk voorwerp in
het universum. Ruimte en tijd, hoewel volkomen onafhankelijk, horen toch bij
elkaar. We kunnen ons immers geen voorstelling maken van een voorwerp
in de ruimte dat gedurende geen tijdsinterval bestaat, en geen
tijdsinterval zonder dat dit plaatsvindt ergens in de ruimte.

\section{Wat is relativiteit?}
Galileo Galilei was de eerste die een principe van relativiteit
formuleerde, hoewel hij dat waarschijnlijk niet erg nauwkeurig
deed. Hij merkte op dat de bemanning van een eenparig bewegend schip
niet kan uitmaken, met experimenten aan boord, wat de snelheid van het
schip is. Zij kunnen de snelheid achterhalen door de relatieve
snelheid van de wal te bepalen, door een voorwerp door het water te
halen, door meting van de snelheid van de wind. Maar er is geen
mogelijkheid de snelheid te bepalen zonder de wereld `buiten het
schip' in ogenschouw te nemen. Een schipper in een kajuit zonder ramen
zal niet eens kunnen bepalen of het schip beweegt of
stilstaat\footnote{Het karakteristieke stampen over de golven
daargelaten! Tegenwoordig gebruiken we vaak treinen of ruimteschepen voor dit soort gedachtenexperimenten}.

Dit is een principe van relativiteit; het zegt dat er geen observeerbare
consequenties zijn van absolute eenparige beweging. Men kan alleen de
snelheid van een voorwerp bepalen {\it ten opzichte van} een ander
voorwerp.

Natuurkundigen zijn empirici; we wijzen een concept af als het geen
observeerbare consequenties heeft. We concluderen dus dat `absolute
beweging' niet bestaat. Objecten hebben alleen een snelheid relatief
aan een ander voorwerp. Elke uitspraak over de snelheid van een
voorwerp moet gemaakt worden ten opzichte van iets anders.

Onze taal is vaak misleidend in het gebruik van `snelheid', omdat we
niet vermelden ten opzichte waarvan. Bijvoorbeeld, een agent kan u
zeggen ``Pardon, weet u wel dat u met een snelheid van 140 kilometer
per uur reed'', zonder erbij te vermelden ``ten opzichte van de
aarde''. Er wordt impliciet aangenomen dat de snelheid gemeten wordt
ten opzichte van de snelweg; u zult de bekeuring bij de rechter niet
kunnen aanvechten op grond van het relativiteitsprincipe van Galilei.

Toen Kepler zijn heliocentrisch model van het zonnestelsel
introduceerde werd hij tegengewerkt op grond van het `gezonde
verstand'. Als de aarde rond de zon draait, waarom `voelen' we deze
beweging dan niet? Het antwoord volgt uit het principe van
relativiteit: er zijn geen lokale observeerbare consequenties als
gevolg van deze beweging\footnote{Er zijn wel degelijk observeerbare
consequenties als gevolg van de rotatie van de aarde om zijn eigen
as. Denk bijvoorbeeld aan de slinger van Foucault en het bestaan van
wervelwinden. Het gaat hier om het feit dat er geen observeerbare
consequenties zijn als gevolg van de {\it rechtlijnige} beweging van
de aarde door de ruimte.}. Nu dat de beweging van de aarde door de
ruimte algemeen is geaccepteerd, is dit het beste bewijs geworden voor
het relativiteitsprincipe. We zijn ons niet dagelijks bewust van het
feit dat de aarde met een snelheid van ongeveer 30 km s$^{-1}$ ($\sim$100.000
km uur$^{-1}$) om de zon draait, dat de zon met een snelheid van 220
km s$^{-1}$ rond het centrum van het melkwegstelsel draait en dat het
melkwegstelsel zelf met een snelheid van ongeveer 600 km s$^{-1}$,
samen met een cluster van sterrenstelsels, beweegt ten opzichte van de
kosmische achtergrondstraling. We hebben deze bewegingen alleen kunnen
bepalen door observaties buiten respectievelijk de aarde, de zon en
ons melkwegstelsel. Het leven van alledag is consistent met een
stilstaande aarde.

\subsection{Einstein's relativiteitsprincipe}

Het relativiteitsprincipe van Einstein zegt, grofweg, dat elke wet in
de natuurkunde en elke fundamentele natuurconstante hetzelfde is voor
alle niet-versnellende waarnemers; dit geldt met name voor de
voortplantingssnelheid van licht. Dit principe is gebaseerd op de
theorie van het elektromagnetisme zoals uiteindelijk geformuleerd door
Maxwell in 1870. Het relativiteitsprincipe van Einstein is niet anders
dan dat van Galilei. Maar het stelt wel expliciet dat ook
elektromagnetische experimenten de schipper van een boot niet in staat
stelt te vertellen of de boot beweegt of stilstaat. Met andere
woorden, de schipper in de kajuit zonder raam zal met de meting van de
snelheid van het licht niet de snelheid de boot kunnen achterhalen.
Terwijl Galilei nog dacht aan kommen met soep en kanonskogels die
langs de mast naar beneden vallen, is het relativiteitsprincipe van
Einstein een uitbreiding hierop.

De Maxwell-vergelijkingen beschrijven alle elektromagnetische
verschijnselen, zoals wisselwerkingen tussen magneten, het gedrag van
elektrische ladingen en stromen, en ook het verschijnsel licht. Licht
wordt gezien als een golvend elektromagnetisch veld. De vergelijkingen
hangen af van de lichtsnelheid $c$ in vacu\"um\footnote{In de Maxwell
vergelijkingen wordt de lichtsnelheid $c$ gegeven door
$c=1/\sqrt{\varepsilon_0 \mu_0}$, waarbij $\varepsilon_0$ de
permitiviteit van het vacu\"um is en $\mu_0$ de magnetische
permeabiliteit.}, en hiermee hangt ook het gedrag van
elektromagnetische verschijnselen af van de lichtsnelheid $c$. Met
andere woorden: als de snelheid van het licht voor twee waarnemers
verschillend zou zijn, zouden de waarnemers dit kunnen achterhalen
door een experiment uit te voeren met magneten, ladingen en
stromen. Einstein veronderstelde een zeer sterk relativiteitsprincipe,
namelijk dat de eigenschappen van magneten, ladingen en stromen
hetzelfde zijn voor alle waarnemers, ongeacht hun onderlinge
snelheid. Einsteins veronderstelling werd gesterkt door experimentele
metingen uit die tijd\footnote{Het meest beroemde voorbeeld is het
experiment door Michelson \& Morley; we zullen dat verderop
bespreken. Het is echter onduidelijk of Einstein zelf van deze
resultaten wist; hij heeft dat altijd ongewis gelaten.}, en is
ondertussen in vele experimenten bevestigd.

De consequenties van dit principe zijn enorm. Dit college is geheel
toegewijd aan de soms merkwaardige voorspellingen en
tegen-intu\"itieve resultaten die hier een gevolg van zijn. Misschien
wel het meest duidelijke maar moeilijk te accepten voorbeeld is dat de
klassieke regel waarmee snelheden worden opgeteld niet meer juist
blijkt:

Stel een schipper Alex (A) passeert met zijn schip een waarnemer Bart
(B) met snelheid $u$. Als A nu een appel gooit met snelheid $v'$ (ten
opzichte van zichzelf) in de richting waarin het schip beweegt, zal
waarnemer $B$ de appel zien bewegen met een snelheid $v=v'+u$. {\it
Deze regel voor het optellen van snelheden is fout!} Of stel voor dat
A de appel laat vallen in het water en de golven beschouwt die door de
plons veroorzaakt worden. Als B stil staat ten opzichte van het water
en de golven van het water voortbewegen met een snelheid $w$ ten
opzichte van het water, ziet B de golven natuurlijk voortbewegen met
snelheid $w$. Waarnemer A echter, op het schip, ziet deze golven
bewegen met een snelheid $w'=w-u$. {\it Ook deze regel voor het
optellen van snelheden is fout!}

Want stel u voor dat er niet met een appel wordt gegooid, maar
waarnemer A schijnt met licht uit een zaklamp. In dit geval, als we
het relativiteitsprincipe van Galilei aannemen, zijn er twee mogelijke
voorspellingen voor de snelheden waarmee A en B het licht zien
voortplanten uit de zaklamp. Als het licht zich zo gedraagt dat het
met snelheid $c$ uit de zaklamp voortplant, dan ziet waarnemer B het
licht met een snelheid $c+u$ ten opzichte van zichzelf. En in het andere geval, als het
licht zich zo gedraagt dat het met een snelheid $c$ beweegt ten
opzichte van een of ander `medium'\footnote{De 19 eeuwse natuurkunde
had de naam `ether' voor dit medium} (analoog met water voor de
watergolven), dan verwachten we dat A het licht observeert met een
snelheid $c-u$ en B met snelheid $c$ (waarbij we aannemen dat B
stilstaat ten opzichte van dit medium). Het wordt gecompliceerder als
beide waarnemers bewegen ten opzichte van dit medium, maar de
conclusie blijft hetzelfde: beide waarnemers zullen een verschillende
snelheid van het licht observeren als we de klassieke regel gebruiken
voor het optellen van snelheden.

Einsteins relativiteitsprincipe verlangt dat zowel A als B dezelfde
lichtsnelheid observeren; met als gevolg dat de optelformule voor
snelheden moet worden aangepast. Uit talloze experimenten is
gebleken dat de versie van Einstein de juiste is. We zullen een nieuwe,
correcte intu\"itie moeten opbouwen, gebaseerd op het
relativiteitsprincipe van Einstein.











%%% Local Variables: 
%%%mode: latex %%% TeX-master: "Inleiding" %%% End:
