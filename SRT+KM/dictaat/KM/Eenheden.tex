\chapter{Eenheden en dimensies}

In dit hoofdstuk(je) worden belangrijke afspraken gemaakt zodat we later
natuurkundige grootheden in een zinnige vorm kunnen gieten. 
Stof uit Giancoli:
\begin{itemize}
\item Hoofdstuk 1.4 en 1.7
\end{itemize}

\section{Het Systeme International}

Bij elke grootheid die je uitrekent of meet moet je een afspraak maken in wat
voor {\bf eenheid} je de meting gaat presenteren. Een meting van een gewicht
van "5" is behoorlijk zinloos als je er niet bij vertelt of we te maken hebben met
grammen, kilogrammen of tonnen. Op het eerste gezicht zou je kunnen denken
dat het een onbegonnen werk is om allerhande grootheden die je in de natuur
tegenkomt uit te drukken in hun eigen eenheid, maar het aardige is nou
dat je de eenheden van maar een paar grootheden hoeft  vast te leggen. Als
je bijvoorbeeld de eenheden van lengte en tijd hebt vastgelegd, dan volgt
automatisch een eenheid voor snelheid als de eenheid van lengte gedeeld
door de eenheid van tijd.

Binnen de klassieke mechanica en ook de speciale relativiteitstheorie heb je 
maar een paar basis eenheden nodig. De eenheden zoals wij ze gebruiken 
zijn vastgelegd in het Systeme International - of kortweg SI systeem:
\begin{itemize}
\item {\bf Lengte.} De lengte van een object wordt gemeten in meter~(m). Een meter
is gedefinieerd als de afstand afgelegd door licht in vacu\"{u}m in 1/299 792 458~seconde.
\item {\bf Tijd.} We meten tijden in seconde, waarbij de seconde~(s) is gedefinieerd
als de duur van 9.192.631.770 perioden van de straling horend bij de overgang tussen
twee energieniveaus van een Cesium atoom.
\item {\bf Massa.} Alle massa's worden gemeten in kilogram~(kg). Een kilogram is 
gelijk aan de massa van het internationale prototype van de kilogram:)
\end{itemize}
Verdere SI basis-eenheden zijn er voor elektrische stroom~(Ampere), temperatuur~(Kelvin)
hoeveelheid materie~(mol) en licht-intensiteit~(candela), maar die komen tijdens
het college klassieke mechanica niet aan bod. Het SI systeem van eenheden wordt
in veruit de meeste landen officieel gebruikt als standaard, met enkele uitzonderingen
waarvan de VS de meest opmerkelijke is. Daar gebruikt men liever inches en pounds.

Alle grootheden die we tegen zullen komen zijn uit te drukken in deze basis SI eenheden.
Als je bijvoorbeeld kijkt naar een kracht, dan weten we dat deze wordt gemeten in Newton
of simpelweg $N$. Aangezien we weten dat $\vec{F}=m\vec{a}$, kunnen we direct zien 
dat de eenheid $N$ uit te drukken is in $\mbox{kg}\cdot\mbox{m}\cdot\mbox{s}^{-2}$. En
dergelijke relaties gelden voor alle grootheden die we tegen zullen komen. Zodra je
een vraagstuk krijgt waarbij er een antwoord in de vorm van een getal uit moet komen
(in plaats van een formule) dan wordt verwacht dat er een eenheid bij dat getal
staat. Anders heeft je antwoord geen betekenis.

\section{Dimensie analyse}

Een krachtig middel om fouten in een berekening op te sporen is het toepassen van 
een dimensie analyse. Als je een formule presenteert weet je dat aan beide kanten
van het "=" teken grootheden staan met dezelfde eenheden. Laten we eens kijken
naar een vergelijking die een verband maakt tussen de positie, $x$ van een object en
zijn snelheid, $v$, als functie van de tijd, $t$. ({\it zoals je kan zien is de relatie evident fout!}):
\begin{equation}\label{eq:dim1}
x =  v\,t^2  
\end{equation}
Laten we aannemen dat we weten dat snelheid de eenheid heeft van lengte gedeeld 
door tijd. Dus:
\begin{equation}
[v]=[L][T]^{-1}
\end{equation}
De vierkante haakjes geven aan dat we het over de dimensie van een grootheid hebben. 
De dimensie van snelheid is dus de dimensie van lengte aangegeven door [L] maal
de dimensie van 1/tijd, $[T]^{-1}$. Als we dit invullen in vgl.~\ref{eq:dim1} een 
redelijkerwijs aannemen dat de dimensie van $x$ wordt gegeven door $[L]$dan krijgen 
we de volgende dimensie vergelijking:
\begin{equation}
[L] \neq [L][T]^{-1}[T]^{2}= [L][T] 
\end{equation}
Aangezien de dimensies aan beide kanten van het gelijkteken niet hetzelfde zijn kan de 
vergelijking dus niet juist zijn. Dit is natuurlijk een lachwekkend eenvoudig geval, maar
een dimensie analyse werkt altijd. Het is trouwens wel zo dat je een beetje moet oppassen
met een dimensie analyse: je kan er alleen mee zien of een antwoord zou kunnen kloppen. 
Als je als antwoord voor plaats als functie van tijd zoals hierboven $x=3v\,t$ had gevonden
zijn de dimensies prima in orde terwijl het antwoord toch niet juist is. 

Het zal duidelijk zijn dat een dimensie analyse alleen mogelijk is als je een antwoord hebt
uitgedrukt in symbolen behorend bij de relevante grootheden in plaats van in gewone getallen. 
Naast het feit dat uitdrukkingen in symbolen veel duidelijker zijn en algemener toepasbaar
zijn, zijn ze dus ook nog eens eenvoudiger op juistheid te controleren. De docent van dit vak
heeft dan ook een sterke voorkeur voor het uitdrukken van antwoorden in symbolen. En de
docenten van andere vakken ook!

\begin{center}
\line(1,0){250}
\end{center}
\begin{voorbeeld} 
\label{ex:dimensie}
Twee studenten hebben bepaald dat voor de slingertijd $t$ van een slinger
de lengte $l$ en de zwaartekrachtsversnelling (met dimensie $[L][T]^{-2}$) de 
relevante grootheden zijn. Student~1 zegt dat de $t=\sqrt{l/g}$ terwijl student~2
stelt dat $t=\sqrt{g/l}$. Wie zou er gelijk kunnen hebben?
{\bf Oplossing: }{\it De dimensie voor het antwoord van student~1 is $\sqrt{[L]/([L]/[T]^2)}=[T]$
en de dimensie voor het antwoord van student~2 is $\sqrt{([L]/[T]^2)/[L]}=[T]^{-1}$. Student~1
zou het dus bij het rechte eind kunnen hebben. Student~2 zit er hoe dan ook naast met
zijn antwoord.
 }
\end{voorbeeld}
\begin{center}
\line(1,0){250}
\end{center}

Een andere plek waar een dimensie analyse handig is, is wanneer je een antwoord 
vindt op een vraagstuk in termen van sinussen, cosinussen, e-machten of andere functies.
Het argument van een dergelijke functie {\it kan} alleen maar juist zijn indien het geen dimensie
heeft.  Laten we de e-macht, $\exp x$  maar eens als voorbeeld nemen. Voor kleine
waarden van $x$ kan deze functie benaderd worden door:
\begin{equation}
\exp x \approx 1 + x 
\end{equation}
Als $x$ nu een dimensie heeft bijvoorbeeld $[x] = [L]$ dan zie je aan de benadering van
de e-macht  dat je een stuk zonder dimensie "1" optelt bij een grootheid met dimensie $[L]$. Dat
kan natuurlijk nooit een fysisch juist antwoord opleveren.  Voor sinussen, cosinussen en 
vele andere functies bestaan ook dergelijke benaderingen en dus gelden daarvoor precies
dezelfde argumenten als het gaat om dimensie analyse.

\section{Wat moet ik weten en kunnen?}

Je moet na dit introducerende hoofdstuk weten:
\begin{itemize}
\item Wat is het SI systeem.
\item Wat zijn de eenheden van massa, lengte en tijd.
\end{itemize}
En wat moet je kunnen:
\begin{itemize}
\item Eenheden van veel voorkomende grootheden zelf bepalen.
\item Een dimensie analyse uitvoeren om te controleren of je een 'geloofwaardig' antwoord
op een vraagstuk hebt gevonden. 
\end{itemize}

\section{Opdrachten}

\begin{enumerate}
\item Opgaven uit Giancoli hoofdstuk~1: 35 t/m 38
\item Als de dimensieanalyse aangeeft dat je antwoord niet fout hoeft te zijn, waarom betekent
dat niet automatisch dat je antwoord goed is?
\end{enumerate}

%%%%%%%%%%%%%%%%%%%%%%%%%%%%%%%%%%%%%%%%%%%%%%%
% EINDE EENHEDEN en DIMENSIES
%%%%%%%%%%%%%%%%%%%%%%%%%%%%%%%%%%%%%%%%%%%%%%%
