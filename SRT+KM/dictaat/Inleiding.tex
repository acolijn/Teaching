\chapter*{Inleiding}
\addcontentsline{toc}{chapter}{Inleiding}

\section*{Waar hebben we het over?}

Het college klassieke mechanica en speciale relativiteitstheorie behandelt in parallel twee belangrijke 
onderwerpen uit de natuurkunde. Aan de ene kant komt de klassieke natuurkunde aan bod, waarin de
basis wordt gelegd voor in principe \emph{alle} vervolgvakken in het curriculum. Er wordt bekeken hoe
het 'bewegen' van objecten nou precies werkt. Hiervoor gaan we de drie wetten van Newton bestuderen
die het hart vormen van de klassieke mechanica. Belangrijke begrippen als eenparige beweging, 
eenparig versnelde beweging, energie en impuls worden behandeld. Deze begrippen zijn zo algemeen
dat ze ook buiten de klassieke mechanica van essentieel belang zijn. Het heeft eigenlijk geen zin om 
geavanceerde vakken als quantummechanica te bestuderen zonder een degelijk begrip van de klassieke
mechanica. 

De andere helft van dit vak behandelt de speciale relativiteitstheorie van Einstein. In dit vak gaan we over
de grenzen van de klassieke mechanica en maken we een paar gewaagde aannames. Redenaties op 
basis van deze aannames leidt tot conclusies over ruimte en tijd die in perfecte tegenspraak zijn met
ons gevoel hoe de natuur 'zou moeten werken'. We gaan onderzoeken of de debiele conclusies die
we moeten trekken zijn gerechtvaardigd en ook zullen we de samenhang tussen de relativiteitstheorie en
de klassieke mechanica onderzoeken.

\section*{Wat wordt er van jullie verwacht?}

In het algemeen zijn de vakken die worden onderwezen tijdens het 1e jaar van de bachelor zogenaamde 
stapelvakken: zonder kennis van het ene  vak is het moeilijk een begin te maken met het volgende. En zelfs
binnen een vak is het vaak zo dat het missen van \'{e}\'{e}n college het volgen van de rest verhindert. Er 
wordt daarom van jullie verwacht dat je actief meedoet aan liefst alle colleges en werkcolleges. In de
college's wordt de stof behandeld die je moet weten, en worden er enkele voorbeelden uitgewerkt. Tijdens
een werkcollege moet je zelf aan de slag. Met name het maken van opgaven is erg lastig in het begin, omdat
je soms niet echt een gevoel hebt hoe en waar je aan natuurkundige problemen moet beginnen.  Enig 
doorzettingsvermogen is handig op dit punt. 

Naast de contacturen wordt er van jullie verwacht dat je werkt aan het college. De richtlijn is dat je aan 'zelfstudie'
ongeveer evenveel tijd besteedt als aan colleges+werkcolleges. Dit klinkt als veel tijd en dat is het ook. Het
blijkt soms moeilijk om je hiervoor te motiveren, maar de docenten gaan er wel vanuit dat deze tijd ook echt
in de colleges wordt geinvesteerd door jullie.

\section*{Wat is lastig?}

Er zijn een paar valkuilen waar je in kan trappen tijdens dit college en waardoor je het jezelf extra lastig kan 
maken. Sommige van deze valkuilen horen specifiek bit dit vak, en  andere valkuilen zijn meer algemeen
geldig voor de studie natuurkunde. Bij deze een selectie gemaakt door jullie docenten:
\begin{itemize}
\item{\bf Klassieke mechanica ken ik al van het VWO.}  Vaak gehoorde klacht is dat je alles al zou kennen. Het
is natuurlijk deels waar dat je een deel van de klassieke mechanica al eens eerder gezien hebt op het VWO, maar
tijdens de natuurkunde studie zal je voor het eerste de interne samenhang van de theorie onderzoeken. Bovendien
is het niveau van het college veel abstracter en gaan we dieper op de stof in.
\item{\bf Tijd en zelfstudie.}  Het blijkt lastig om zelf aan vakken te werken buiten de colleges om. Moet toch, maar
vraagt discipline.
\item{\bf Het rekenbeest.} Bijna niet nodig.
\item{\bf Formules en getallen.} Gerelateerd aan het vorige punt. We rekenen vaak formules uit, en we vragen
lang niet altijd om een numeriek resultaat. Het grote voordeel van het werken en rekenen met formules is dat je
antwoorden krijgt die algemeen geldig zijn ipv voor een enkel voorbeeld.  
\end{itemize}

\section*{Opzet van het college}

Zoals hierboven al genoemd volgt dit college de klassieke opzet van een hoorcollege in combinatie met 
werkcolleges.  Per week worden er $2\times2$ uur college gegeven, met een werkcollege bij ieder
hoorcollege. De werkcolleges gebeuren in een kleinere setting, waarbij je in een groep van ongeveer
25 studenten wordt geholpen door je "eigen" assistent. De toetsing van het college gebeurt door
middel van een tentamen aan het eind.  Daarnaast is er halverwege de cursus een deeltentamen en wordt
er van jullie verwacht dat er iedere week digitale opgaven worden gemaakt.



