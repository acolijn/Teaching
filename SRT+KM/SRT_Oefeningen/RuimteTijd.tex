\section{Tijddilatatie en lengtecontractie}



\subsection{Einstein puzzel}
Einstein, als jongen van 16, vroeg zich het volgende af: een hardloopster
ziet zichzelf in een spiegel die zij in haar hand houdt, een armlengte
voor haar gezicht. Als zij nu met bijna de snelheid van het licht
rent, zal ze zichzelf dan nog steeds in de spiegel zien? Analyseer deze
vraag aan de hand van het relativiteitsprinciepe.

\subsection{Tijddilatatie}
Een waarnemer D heeft een lichtklok en een polshorloge, beide in rust
ten opzichte van hemzelf.  Een andere waarnemer E beweegt ten opzichte
van waarnemer D, en kan de lichtklok en polshorloge van D bekijken.
We vragen ons af of het mogelijk is dat waarnemer D zijn twee klokken
gelijk ziet lopen, terwijl waarnemer E observeert dat de klokken niet
gelijk lopen. Laat met behulp van een gedachtenexperiment zien dat dit
onmogelijk is \textit{(hint: stel voor dat de beide klokken van D bij elke tik
een gaatje prikken in een tape)}.


\subsection{Einsteins \textit{Gedankenexperiment}}
Als werknemer bij een patentburo in Bazel zag Einstein vanaf zijn
werkkamer de klok van de kerktoren. Op een bepaald moment stond de
klok op precies 3~uur.  Hij stelde zich voor dat het licht, dat
weerkaatst wordt vanaf de klok en het `beeld' van de klok met zich
draagt, met een snelheid van 300.000 km/s de ruimte in suist. Hij vroeg
zich af hoe je de klok ziet lopen als je met dit licht zou kunnen
meereizen.
\begin{itemize}
\item [a.] Hoe verloopt de tijd voor een (hypothetische) waarnemer die met de snelheid van het licht reist?
\end{itemize}

\subsection{Afstanden}
In het ruststelsel van de aarde is de afstand tussen Amsterdam en New
York ongeveer 6000 km (5.877 km om precies te zijn). Met hoeveel wordt
de afstand tussen de steden verkort, geobserveerd vanuit een
vliegtuig (1000 km/uur)? Of door de International Space Station (8
km/s)? Of door een kosmisch deeltje dat met een snelheid van 0,9$c$
reist?

%%%%%%%
\subsection{Muon}
Een muon ($\mu$-deeltje) is een instabiel elementair deeltje, dat in 
{\it rust} een gemiddelde levensduur $\tau_{0} = 2,2 \cdot 10^{-6}$ s  heeft.  
Veronderstel dat een gegeven muon in een laboratorium (stelsel $S$) een 
buis met een lengte $l =600$ m kan doorlopen voordat het vervalt. 
\begin{itemize}
\item [a.]
  Druk de levensduur $\tau$ van een muon in het laboratorium uit in zijn 
snelheid $v$.
\item [b.]
  Gebruik het gegeven dat het muon binnen $\tau$ s de buis van $600 $ m kan 
doorlopen om $v$ te berekenen. 
\item [c.]
  Hoe lang was volgens het muon zelf de buis die aan hem voorbij schoot? 
\item [d.]
  Leeft het muon volgens zichzelf lang genoeg om de Lorentz-gecontraheerde 
buis te passeren?
\end{itemize}

%%%%%%%%
\subsection{Neutron}
De gemiddelde levensduur van een neutron is 15 minuten 
(daarna vervalt hij in een proton, elektron en antineutrino). 
Toch zijn er neutronen die vanuit de zon de aarde bereiken (afstand 
$\approx 1,5 \cdot 10^{11}$  m). 
  Met welke snelheid moeten die minstens door de zon zijn uitgestoten?

%%%%%%%%%%
\subsection{Astronaut}
Een astronaut wil binnen \'{e}\'{e}n jaar (volgens zijn eigen tijdrekening) 
een ster die op een afstand van 5 lichtjaren staat bereiken. 
Neem als lengte-eenheid lichtjaar en als tijdseenheid jaar.
\begin{itemize}
\item [a.]
Welke waarde heeft de lichtsnelheid $c$ in deze eenheden?
\item [b.]
Welke snelheid moet zijn ruimteschip dan hebben?
\item [c.]
Hoelang duurt de reis volgens de aardse tijdrekening?
\end{itemize}

%%%%%%%%%
\subsection{Bewegend voorwerp}
\begin{itemize}
\item [a.]
  Hoe verandert door de Lorentz-contractie de vorm en de inhoud van een 
bewegend volume?
\item [b.]
  Hoe verandert de dichtheid van een bewegend voorwerp?
\end{itemize}

