
%%%%%%%%
%pretest\subsection{electronvolt}
%pretest\begin{itemize}
%pretest\item [a.]
%pretestWat is de definitie van electronvolt?
%pretest\item [b.]
%pretestHoeveel Joule is 1 GeV?
%pretest\item [c.]
%pretestGeef de formule voor de energie van een relativistich deeltje.
%pretest\item [d.]
%pretestDe massa van een electron is $m_{e} = 0,511$ MeV/c$^{2}$.
%pretestWat is de snelheid van een elektron dat een energie van 1 GeV heeft?
%pretest\end{itemize}

%%%%%%%%%%%
%\subsection{Massieve deeltjes}
%Niet-relativistisch heeft een vrij deeltje met massa $m$ en 
%snelheid $v$ een energie $E = \frac{1}{2}mv^{2} = \frac{p^{2}}{2m}$.
%\begin{itemize}
%\item [a.]
%Wat is $\frac{dE}{dp}$?
%\end{itemize}
%Een relativistisch deeltje met massa $m$ en snelheid $v$ heeft een energie
%$E = \sqrt{m^{2}c^{4} + c^{2}p^{2}}$.
%\begin{itemize}
%\item [b.]
%Wat is nu $\frac{dE}{dp}$?
%\item [c.]
%Laat zien waarom het deeltje niet sneller kan bewegen dan de lichtsnelheid.
%\end{itemize}
 

\section{Lorentztransformatie van impuls en energie}

%%%%%%%%
\subsection{Impuls-energie viervector}\label{Ievv}
De componenten van de impulsvector en de energie vormen een vier-vector
$(\frac{E}{c}, p_{x}, p_{y}, p_{z})$ op dezelfde manier als de 
tijd-ruimte-co\"{o}rdinaten $(ct, x, y, z)$.
Bij overgang van stelsel $S$ naar $S'$ transformeren ze volgens dezelfde 
Lorentztransformatie als voor $(x, y, z, ct)$ geldt.
\begin{itemize}
\item [a.]
Vertaal de transformatie 
$x' = \gamma (x - \beta ct),\ ct' = \gamma (ct - \beta x)$ naar
$p_{x}$ en $\frac{E}{c}$.
\end{itemize}
%Als een deeltje stil ligt in $S'$, beweegt het met snelheid $v$ in $S$.
We bekijken een deeltje dat door stelsel $S$ beweegt met snelheid $v$; dit deeltje ligt dus stil in zijn eigen `rustframe' $S'$.
\begin{itemize}
\item [b.]
Laat zien dat je met de Lorentztransformatie voor de impuls en energie van het
bewegende deeltje in $S$ de volgende formules krijgt:
\begin{eqnarray*}
p_{x} &=& \gamma mv\\
E & = & \gamma mc^{2}
\end{eqnarray*}
\end{itemize}
Bij de Lorentztransformatie blijft de combinatie $(ct)^{2} - x^{2}$
hetzelfde, \textit{Lorentzinvariant}.
\begin{itemize}
\item [c.]
Controleer dat voor $(ct)^{2} - x^{2}$ en $(ct')^{2} - x'^{2}$.
\item [d.]
Welke Lorentzinvariante combinatie van $E$ en $p_{x}$ neemt de plaats in van $(ct)^{2} - x^{2}$?
\item [e.]
Welke waarde neemt deze combinatie aan in het rustframe $S'$?
\end{itemize}

%%%%%%%%
\subsection{Energie van een relativistisch deeltje}
Voor een relativistisch deeltje is het verband tussen energie en impuls:
\begin{displaymath}
E^{2} - c^{2}p^{2} = m^{2}c^{4}
\end{displaymath}
dus:
\begin{displaymath}
E = \sqrt{m^{2}c^{4} + c^{2}p^{2}}
\end{displaymath}
\begin{itemize}
\item [a.]
Teken de grafiek van de energie van het bewegende deeltje als functie van 
zijn snelheid $v$.
Geef hierin de bijdrage van de kinetische energie $E_{k} = E - E_{0}$
aan ($E_{0}$ is de rustenergie van het deeltje).
\item [b.]
Hoe snel moet een deeltje bewegen om zijn kinetische energie even groot 
te laten zijn als zijn rustenergie?
\item [c.]
Tot welke uitdrukking reduceert de kinetische energie voor niet-relativistische
deeltjes ($pc \ll E_{0}$)?
\item [d.]
Dezelfde vraag voor relativistische deeltjes ($pc \gg E_{0}$).
\end{itemize}

%%%%%%%
%\subsection{Massa van een relativistisch deeltje}
%Een deeltje met massa $m$ beweegt met snelheid $v = 0,40$c.
%\begin{itemize}
%\item [a.]
%Als je de snelheid van het deeltje verdubbelt, verdubbelt dan ook de impuls 
%van het deeltje zoals je niet-relativistisch zou verwachten?
%Controleer met een berekening.
%\item [b.]
%Als je de impuls van het deeltje verdubbelt, wordt dan de
%kinetische energie van het deeltje vier keer zo groot (zoals in de 
%niet-relativistische mechanica)?
%\end{itemize}

%%%%%%%%%%
\subsection{Energie en impuls van een relativistisch deeltje}
Als je aan een deeltje energie toevoegt, nemen de snelheid en impuls van het 
deeltje toe.
\begin{itemize}
\item  [a.]
Tot welke waarde nadert de snelheid van het deeltje uiteindelijk?
\item  [b.]
En de impuls?
\item  [c.]
Teken de grafiek van de energie (verticaal) tegen de impuls 
(neem $cp$, horizontaal).
Geef daarin ook de grafiek van de niet-relativistische energie 
$E = E_{0} + \frac{p^{2}}{2m}$.
\item [d.]
Tot welke rechte lijn nadert de relativistische grafiek uiteindelijk?
\end{itemize}

%%%%%%%%%%%
\subsection{Massaloze deeltjes}
Fotonen zijn massaloze deeltjes.
\begin{itemize}
\item [a.]
%Waarom is hun snelheid gelijk aan $c$?
Laat met de formules uit opdracht \ref{Ievv}b. zien dat voor een deeltje met massa geldt:

\begin{equation}
	v=\frac{pc^2}{E} \nonumber
\end{equation}

\item [b.]
%Wat is de formule voor de energie van een foton?
Laat zien dat voor massaloze deeltjes, zoals fotonen geldt:

\begin{equation}
	E=pc \nonumber
\end{equation}

Waarom volgt dan dat $v=c$?

\end{itemize}
De energie van een foton hangt af van de frequentie: $E_{foton} = h\nu$ 
($\nu$ is de frequentie en $h$ is de constante van Planck).
\begin{itemize}
\item [c.]
Hoe hangt de impuls van een foton af van zijn golflengte?
Gebruik $c = \lambda\nu$, waarbij $\lambda$ de golflengte is.
\item [d.]
Als je op het strand in de zon ligt, wordt je bestraald met een sterkte van 
ongeveer 100 W.
Met welke impuls botst deze straling elke seconde tegen je aan (de 
"stralingsdruk")?
\end{itemize}

%%%%%%%%%%%%%%%%
\subsection{Twee relativistische deeltjes}
In stelsel $S$ bewegen twee identieke deeltjes $A$ en $B$ langs
de $x$-as naar elkaar toe, elk met een snelheid van $0,8c$.
Dus $v_{A} = +0,8c$ naar rechts en $v_{B}$~$=$~$-0,8c$ naar links.
De massa van de deeltjes is $m$.
\begin{itemize}
\item [a.]
Druk in $S$ de impuls en de energie van de deeltjes uit in $m$ en 
$c$.
\item [b.]
Hoe groot is de totale kinetische energie uitgedrukt in $m$ en $c$?
\end{itemize}
We bekijken de situatie nu vanuit stelsel $S'$, dat met $A$ meebeweegt. 
In $S'$ staat $A$ dus stil en komt $B$ met een extra grote (negatieve)
snelheid op $A$ af.
\begin{itemize}
\item [c.]
Schrijf de Lorentztransformaties voor energie en impuls voor de overgang van 
$S$ naar $S'$ op.
\item [d.]
Vul de getalswaarden voor $\gamma$ en $\beta$ in in de Lorentztransformaties en
bepaal de impuls en energie in $S'$ voor deeltje $A$. 
Was de uitkomst te verwachten?
\item [e.]
Bepaal met de Lorentztransformaties ook de impuls en energie van 
deeltje $B$ in $S'$.
\item [f.]
Hoe groot is volgens de `optelformule' voor snelheden de snelheid
van $B$ in $S'$?
\item [g.]
Geven de formules $p = \gamma mv$ en $E = \gamma mc^{2}$ voor het
bewegende deeltje $B$ in $S'$ dezelfde antwoorden als in vraag e).
\end{itemize}

\newpage
%%%%%%%%%%%%%%
\subsection{Inelastische boting}
Twee identieke deeltjes met massa $m$ worden op elkaar afgeschoten.
De \'{e}\'{e}n heeft een snelheid $\frac{3}{5}c$ naar rechts, 
de ander een snelheid $\frac{3}{5}c$ naar links.
Na de botsing zijn ze versmolten tot \'{e}\'{e}n deeltje met massa $M$.
\begin{itemize}
\item [a.]
Waarom staat het deeltje met massa $M$ na de botsing stil?
\item [b.]
Wat verwacht je voor de massa $M$:
\begin{itemize}
\item [1.]
$M = 2m$
\item [2.]
$M > 2m$
\item [3.]
$M < 2m$
\end{itemize}
Licht je keuze toe.
\item [c.]
Hoe groot was de totale energie van de twee deeltjes voor de botsing?
\item [d.]
Deze energie gaat over in de rustenergie van het gecombineerde deeltje, 
hoe groot is $M$?
\end{itemize}

%%%%%%%%%%%%
\subsection{Nog een inelastische botsing}
Een deeltje met massa $m$ botst met een energie $E = 2mc^{2}$ op 
een zelfde deeltje in rust.
\begin{itemize}
\item [a.]
Wat is de snelheid van ieder van de deeltjes?
\item [b.]
Wat is de impuls van ieder van de deeltjes?
\end{itemize}
Na de botsing vormen de twee deeltjes een deeltje met massa $M$ en snelheid
$V$. 
De energie en impuls van deeltje $M$ zijn respectievelijk
$E = \gamma Mc^{2}$ en $p = M\gamma V$.
\begin{itemize}
\item [c.]
Wat is de energie van deeltje $M$ uitgedrukt in de massa van de twee
botsende deeltjes?
\item [d.]
En de impuls?
\item [e.]
Bereken de snelheid $V$ van deeltje $M$.
\item [f.]
Bereken massa $M$.
\item [g.]
Niet-relativistisch zou je verwachten dat $M = 2m$ en $V = \frac{1}{2}v$.
Was dat relativistisch ook zo?
\end{itemize}

%%%%%%%%%%%%%%
\subsection{Nog twee deeltjes}

Gezien vanuit stelsel $S$ bewegen twee identieke deeltjes A en B (elk met massa~$m$) langs de $x$-as naar elkaar toe, de een naar rechts met snelheid $\frac{3}{5}c$, de ander naar links met snelheid $-\frac{3}{5}c$.\\
	
	
 \begin{figure} [h]
 \begin{center}
 \mbox{\epsfxsize=10cm\epsffile{oefeningen.pictures/opgv2.eps}}
 \caption{Botsende deeltjes A en B}
 \label{f:BdAB}
 \end{center}
 \end{figure}
	
			\begin{enumerate}
			\item[a.] Hoe groot is in $S$ de factor $\gamma$ voor A en voor B? Bereken de impuls en energie die A en B in stelsel $S$ hebben.
			\item[b.] Ga voor deeltje A na of zijn energie-impuls de goede invariante norm heeft.
			\item[c.] We bekijken de situatie nu vanuit stelsel $S'$ dat met B meebeweegt. Schrijf de Lorentz-transformatie voor de impuls vier-vector van $S$ naar $S'$ op; ga uit van je antwoord bij vraag a. (Vul $\beta$ en $\gamma$ in als breuk en niet als decimaal getal!)
			\item[d.] Bereken met deze Lorentz-transformatie de waarde van de componenten van de energie/impuls viervector van deeltje A in stelsel $S'$: 
			\begin{equation} p_A' = (\frac{E_A'}{c},p_{x(A)}',0,0).  \end{equation}
			 
			\item[e.] Met welke \textit{kinetische} energie botst A op B in $S'$?
			\item[f.] Laat zien dat A volgens de `snelheids-optelformule' in $S'$ een snelheid $\beta' = \frac{15}{17}$ heeft.
			\item[g.] Bereken de energie van A in $S'$ opnieuw, maar nu met de formules 
			
			\begin{eqnarray}
				p_{x(A)}' &=& \gamma' m_A v_A', \\
				E_A' &=& \gamma' m_A c^2. \nonumber
				\label{eq:}
			\end{eqnarray} 
		
		\end{enumerate}

