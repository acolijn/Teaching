\section{Klassieke Mechanica}
Voor de Klassieke Mechanica zullen een aantal opgaven uit het boek
`Analytical Mechanics' van Fowles \& Cassiday worden behandeld. Dit zijn onder meer:
\begin{itemize}
\item{Hoofstuk 1}: 1.1, 1.2, 1.3, 1.5, 1.6, 1.7
\item{Hoofdstuk 2}: 2.1, 2.2, 2.3, 2.4, 2.6
\item{Hoofdstuk 4}: 4.1, 4.2, 4.3, 4.4, 4.5
\item{Hoofdstuk 7}: 7.1, 7.2, 7.4, 7.5
\end{itemize}
Hieronder volgen een aantal extra opgaven.

%%%%%%%%%%%%
\subsection{Botsende deeltjes}
Voor twee botsende deeltjes $A$ en $B$ bestaat de `wet van behoud van impuls':
\begin{displaymath}
   m_{A}v_{1A} + m_{B}v_{1B} = m_{A}v_{2A} + m_{B}v_{2B}
\end{displaymath}
waar $v_{1A}$ en $v_{1B}$ de snelheden van $A$ en $B$ voor de botsing 
en $v_{2A}$ en $v_{2B}$ de snelheden na de botsing zijn.
$S$ en $S'$ zijn twee \textit{niet relativistische} inertiaalsystemen met een onderlinge snelheid $v$.
Bewijs dat wanneer de behoudswet geldt in $S$, deze ook geldt in $S'$.

%%%%%%%%%%%%
\subsection{Massa-middelpunt}
Het  massa-middelpunt van twee deeltjes is een denkbeeldig punt tussen de 
deeltjes in, waarvan de  plaats,  snelheid  en  versnelling  het  gemiddelde 
is van die van de twee deeltjes, als je tenminste  de  grootste  massa  het  
sterkst  meetelt (gewogen gemiddelde).  
Als de massa's van de deeltjes $m_{1}$ en $m_{2}$ zijn, is hun totale
massa $M =  m_{1} + m_{2}$ en geldt voor hun massa-middelpunt:

\begin{eqnarray*}
   x_{M} & = &  \frac{m_{1}}{M}x_{1} + \frac{m_{2}}{M}x_{2} \\
   v_{M} & = &  \frac{m_{1}}{M}v_{1} + \frac{m_{2}}{M}v_{2} \\
   a_{M} & = &  \frac{m_{1}}{M}a_{1} + \frac{m_{2}}{M}a_{2} 
\end{eqnarray*}

Bij  twee  botsende  deeltjes,  waarop verder geen krachten werken, beweegt 
het massa-middelpunt altijd eenparig.
Je  kunt  dan altijd een Galileitransformatie maken van het 
$L$-systeem (het `laboratorium-systeem') waarin de botsing plaats heeft, 
naar het 
$M$-systeem (het `massamiddelpunt-system')\footnote{In het Engels heet dit het `Centre of mass system', ofwel CM-system.}.

Deeltje $A$ heeft een massa $m_{A} = 4$ kg en botst met een snelheid
$v_{A} = 10 $ m/s op een stilstaand 
deeltje $B$  met massa $m_{B} = 1 $ kg. 
\begin{itemize}
\item [a.]
Laat zien dat de totale impuls in het $M$-systeem
v\'{o}\'{o}r  de  botsing nul is:
$p_{A,M} + p_{B,M} = 0$. 
\end{itemize}

In het $M$-systeem zijn (en blijven) de twee impulsen dus {\it even groot} en 
{\it tegengesteld} aan elkaar!

Bij  een  volkomen  elastische  botsing  gaat  er  geen kinetische energie 
verloren.
Het is dan gemakkelijk  te  bewijzen  dat  in het $M$-systeem de impulsen van 
$A$ en $B$ na de botsing niet alleen even groot en tegengesteld zijn, maar 
ook dezelfde grootte hebben als v\'{o}\'{o}r de botsing.
\begin{itemize}
\item [b.]
Als deeltje $A$ bij een volkomen elastische botsing in het $M$-systeem 
$90^{o}$ zou afbuigen, wat zijn dan de snelheden na de botsing 
in het $M$-systeem? 
\item[c.] En in het $L$-systeem? 
\end{itemize}

%%%%%%%%%%%%%
%\subsection{Krachtveld}
% \begin{figure} [h]
% \begin{center}
% \mbox{\epsfxsize=8cm\epsffile{oefeningen.pictures/Krachten.ps}}
% \caption{Krachtveld}
% \label{f:kracht}
% \end{center}
% \end{figure}
%In het $(x,y)$-vlak is de potentiaalfunctie $V(x,y)=x^4y^2$ gedefinieerd. In de figuur zie je een paar lijnen van constante $V$: op de kromme lijnen is $V=1$ en op de $x$- en $y$-as is $V=0$.
%\begin{itemize}
%\item[a.] 
%Bepaal de krachtcomponenten $F_x$ en $F_y$ voor een willekeurig punt $(x,y)$.
%\item[b.]
%Teken de $\vec{F}$-pijl in punt $B$.
%\item[c.]
%Bepaal van dit krachtveld de rotatie: $\mbox{rot} ~\vec{F} \equiv \nabla \times \vec{F}$.
%\end{itemize}



 